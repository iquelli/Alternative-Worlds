\documentclass[a4paper, 12pt]{article}
\usepackage[hmargin=2cm,vmargin=2.5cm]{geometry}

\usepackage{parskip}
\setlength{\parskip}{1em} % Adjust the length as needed
\usepackage{enumerate}
\usepackage{makecell}
\usepackage{graphicx}
\usepackage{hyperref}
\usepackage{caption}
\usepackage{amsmath}
\usepackage[backend=biber,style=apa, url=true, sortcites]{biblatex}
\usepackage[table]{xcolor}

\hypersetup{
    colorlinks,
    citecolor=black,
    filecolor=black,
    linkcolor=black,
    urlcolor=black
}


\setlength{\arrayrulewidth}{0.4mm}

\newcommand{\HRule}{\rule{\linewidth}{0.5mm}}

\begin{document}

\renewcommand{\figurename}{Figura}
\renewcommand{\contentsname}{Índice}

\begin{titlepage}
  \begin{center}

    \includegraphics[width=0.5\textwidth]{assets/IST-logo.png}~\\[2cm]

    \textsc{\Large Grupo 1}\\[0.5cm]
    \textsc{\Large Mundos Alternativos}\\[0.5cm]
    \textsc{\Large 1º Período 2023}\\[3cm]


    \HRule \\[0.4cm]
    {\large \bfseries Relatório Final \\ CRISPR kit \\[0.4cm]}
    \HRule \\[3cm]

    \large\textbf{Trabalho realizado por:}\\
    \begin{center}
      Duarte Gomes - 103975 \\  Isabela Teixeira - 102542 \\ Lara Gouveia - 104113 \\ Raquel Braunschweig - 102624\\[3cm]
    \end{center}


    \vfill

    {\large 15 de Outubro de 2023}

  \end{center}
\end{titlepage}

  \newpage

  \tableofcontents

  \newpage

  \section{Introdução}

  Num mundo onde a ciência e a tecnologia avançam constantemente, o tema das mutações genéticas é cada vez mais debatido. Imagine um casal que queira procriar conseguir personalizar as características do seu futuro filho. Desde a cor dos olhos até à resistência genética a certas doenças. Este cenário levanta várias questões importantes....

  \begin{center}
    \textbf{  Será que este cenário é assim tão irrealista ou não estará muito longe da realidade? Quais serão as suas implicações éticas?}
  \end{center}

  Atualmente na sociedade, estima-se que cerca de 5\% da população mundial esteja afetada por uma doença genética, olhando para estes dados percebemos rapidamente que a genética desempenha claramente um papel de destaque na compreensão de saúde humana, e que fez com que houvesse um grande desenvolvimento na ciência. Doenças deste género ocorrem de forma espontânea, devido a erros durante a replicação de DNA e exigem cada vez mais atenção por parte da sociedade, tornando um desafio a tentativa de encontrar uma “cura”.
  
  Com o desenvolvimento da ciência, foram desenvolvidas enumeras técnicas de modo a tentar combater este tipo de doenças, sendo a mais revolucionário, o CRISPR. Esta técnica consiste numa edição genética que permite a modificação de DNA e tem origem num sistema imunológico que foi encontrado em bactérias e arqueias, que as ajuda a combater contra vírus invasores.

  \section{Desenvolvimento do Objeto}

  \section{A nossa Instituição}

  Em Novembro de 2018, o cientista chines He Jiankui, na altura com 34 anos, surpreendeu a comunidade científica e o mundo após anunciar numa convenção de Edição de Genomas Humanos os primeiros bebés, duas gémeas, com um genoma alterado. He, alegou que foi utilizada a técnica do CRISPR e que efetuou esta mudança nos embriões, fazendo assim com que o gene de HIV fosse inibido à nascença.
  
  A comunidade científica, inicialmente, considerou o suposto experimento antiético, irresponsável e ilegal (sendo que o CRISPR ainda estava em fase experimental), levando mesmo com que He fosse expulso da faculdade onde trabalhava, condenado a pagar uma multa e a cumprir três anos de prisão.
  
  Durante largos anos este foi um tema bastante discutido, será ético modificar a nossa espécie desta forma? Se alguma modificação corre mal, como vamos fazer para a corrigir sendo que vai afetar todas as gerações futuras? Será que deve ser permitido fazer modificações a embriões?
  
  As gémeas foram submetidas a análises durante vários meses e realmente a experiência de He tinha funcionado, encontravam-se bem de saúde e realmente as análises indicavam imunidade à infeção de HIV. 
  
  Após sair da prisão, He foi contactado por várias organizações com bastante poder, tanto a nível político como a nível financeiro, mas este decidiu recusar e avançar com a sua própria instituição, designada por C-Lab, sabendo que apenas assim iria conseguir ter controlo sobre o desenvolvimento do seu projeto.
  
  Esta instituição começou por ser privada, com o objetivo de tratar e estudar doenças com origem genética em “crianças e adultos, mas não embriões”, de modo a tentar evitar conflitos com algumas questões éticas. O foco inicial começou por ser compreender e explorar as complexidades do genoma humano, de modo que fosse possível aplicar o CRISPR de forma segura e responsável.
  
  Após alguns anos, as técnicas foram cada vez mais aperfeiçoadas e inúmeros testes foram bem-sucedidos quando utilizados em animais, o que levou a ser considerado a possibilidade de ser aplicado em seres humanos. 
  
  A C-Lab começou a ser cada vez mais falada a nível mundial e no ano de 2030, a China quis tornar legal a mutação genética através do método CRISPR, investindo uma quantia bastante elevada na fundação, dando não só a possibilidade de ter mais recursos para continuar a evolução deste projeto, como também ajudando a fundação a crescer e a conseguir espalhar-se por todo o país. 
  
  Foi uma grande conquista para a ciência, tinham conquistado a confiança de um dos maiores países do mundo. Nos primeiros anos, apenas pessoas com doenças crónicas mortais tinham ajuda do governo para os tratamentos e para o acompanhamento necessário. Mais tarde, devido à evolução do tratamento começou a ser mais acessível para todas as pessoas, havendo melhorias bastante significativas no número de mortalidades do país.
 
  Com isto, vários países tornaram-se mais suscetíveis à possibilidade de legalização das modificações genéticas, expandindo-se para a Europa e mais tarde para os Estados Unidos da América. 

  \section{Utilização do Objeto}

  \section{Evolução da Sociedade}
  
  He Jiankui viu todo o mundo à sua volta a mudar. A Terra que ele conhecia, já não era a mesma.

  \subsection{GeneShifters vs PurityKeepers}

  Surgiram dois grupos de pessoas: os \textit{GeneShifters} e os \textit{Purity Keepers}, que muito lhe lembravam os antigos debates entre 
   os \textit{Pro-Life} versus os \textit{Pro-Choice}. No entanto, enquanto uns discutiam leis de interrupção voluntária de gravidez, outros 
   debatiam a manipulação genética.

  Esta divisão era evidente no dia-a-dia. Enquanto que os \textit{GeneShifters} optavam por se adaptar a novas tendências de moda que refletiam não apenas um novo estilo de vida avançado,
   mas como um símbolo de otimismo para o futuro, os Purity Keepers escolhiam permanecer fieis a tradições passadas. Para eles, era uma declaração de resistência a todas as mudanças que
   estavam a assistir. 

  \begin{minipage}{0.5\textwidth}
    \centering
    \includegraphics[width=0.8\linewidth]{assets/GeneShifters.png}
    \captionof{figure}{Moda comum entre \textit{GeneShifters}}
  \end{minipage}%
  \begin{minipage}{0.5\textwidth}
    \centering
    \includegraphics[width=0.8\linewidth]{assets/PurityKeepers.jpg}
    \captionof{figure}{Vestuário dos \textit{PurityKeepers}}
  \end{minipage}

  Enquanto o mundo permanecia dividido, Jiankui estava determinado a ser uma voz de liderança para os \textit{GeneShifters}. Acreditava na 
   necessidade de evoluir. Para ele, este novo kit CRISPR que desenvolveu, apesar de ser uma inovação controvérsia, era uma nova fronteira do conhecimento humano. 
   Uma ferramenta poderosa para melhorar a saúde e a qualidade de vida de futuras gerações.

  Olhava para a resistência dos \textit{Purity Keepers} como nada senão um medo do desconhecido e uma reluctância em abandonar velhas trandições. 
   "Não brinquem de Deus!", "Será que há progresso quando é à custa do nosso legado genético?" - eram algumas das frases que via em cartazes quando passavam pela televisão
   vídeos das diversas manifestações que ocorriam pelo mundo em torno deste tópico.

  \subsection{Transformação do Ambiente}

  As pessoas de maior influência, como os políticos e os bilionários, viram na manipulação genética uma imensa fonte de rendimento. Assim, apesar da firme 
   resistência dos \textit{Purity Keepers}, as suas vozes foram silenciadas, não pelo contragolpe do grupo oposto, mas pela voz mais alta do mundo - a voz da ganância.

  Com a evolução do kit os seres humanos acabaram por poder ser geneticamente modificados para terem uma resistência notável a temperaturas extremas. Indivíduos de maior fortuna muito rapidamente decidiram
   investir quantidades astronómicas de dinheiro em projetos de construção de habitações em locais que nuncam antes tinham presenciado a presença humana. Queriam ter casas de férias nessas
   zonas. E, para isso, adaptaram-se geneticamente não só a si mesmos, mas como aos seus empregados.

  Deste modo, a Antártica, outrora uma vastidão de gelo e neve, foi transformada numa paisagem urbana. O deserto Sahara, anteriormente um mar de areia inexplorado, agora abrigava
   pequenas cidades vibrantes.
  
  \begin{minipage}{0.5\textwidth}
    \centering
    \includegraphics[width=0.8\linewidth]{assets/Antártica.png}
    \captionof{figure}{Antártica}
  \end{minipage}%
  \begin{minipage}{0.5\textwidth}
    \centering
    \includegraphics[width=0.8\linewidth]{assets/Deserto_Sahara.png}
    \captionof{figure}{Deserto Sahara}
  \end{minipage}

  \subsection{Impacto na saúde, fauna e flora}

  Não podemos deixar de mencionar o impacto monumental na área da saúde, que era motivo de grande orgulho para He Jiankui. Doenças genéticas que antes assombravam gerações foram erradicadas, os tratamentos médicos
   tornaram-se altamente personalizados, elevando a eficácia dos medicamentos a patamares nunca antes vistos. Este progresso revolucionário contribuiu para um aumento significativo na esperança da vida humana.

  Além dos humanos, a fauna e a flora também foram alvo de meticulosas modificações genéticas. Sempre que surgia uma nova doença transmitida pelos animais, o governo respondia de forma enérgica, aplicando alterações 
   genéticas para evitar futuras transmissões. Os alimentos, por sua vez, foram aprimorados para resistir a condições adversas, originando colheitas mais resistentes e garantindo uma maior segurança alimentar a escala global.

  \subsection{Legado de He Jiankui}

  É de esperar que após a descoberta revolucionária de He Jiankui a profissão de cientista seja uma das mais cobiçadas do mundo, refletindo o enorme impacto que este aprimorado CRISPR kit teve nas perspectivas e prioridades da
   sociedade.

  Para Jiankui era quase surreal considerar algum domínio que não tivesse sido afetado pela sua grande descoberta. Desde os níveis microscópicos do DNA até às vastas extensões de ecossistemas, o impacto da manipulação 
   genética era evidente. 
   
  Um modesto cientista chinês conduziu a humanidade a um novo capítulo da sua história, repleto de promessas. No entanto, algo que Jiankui inadvertidamente esqueceu foi que 
   cada mudança trazia consigo tanto benefícios quanto desafios, e que nem tudo brilha com a mesma intensidade.

  \section{Impactos Sociais}

  A euforia inicial que envolveu a descoberta de Jiankiu rapidamente deu lugar a uma realidade mais sombria. Para o desgosto do cientista, nem tudo foi um mar de rosas. A manipulação genética, apesar das suas promessas, trouxe consigo uma 
   série de impactos negativos que afetaram profundamente a sociedade.

   \subsection{Desigualdades Sociais}

   Primeiramente, assistiu-se a uma acentuação notável das disparidades socioeconómicas. O kit CRISPR, como referido na secção 3, começou por ser utilizado num instituto privado, só mais tarde tornando-se de domínio público. Esta discrepância 
    na acessibilidade à tecnologia genética aprofundou ainda mais a divisão entre os estratos sociais. Os indivíduos mais abastados, com recursos financeiros substanciais, tiveram acesso facilitado e precoce às potenciais vantagens 
    da manipulação genética. Enquanto isso, aqueles de recursos mais limitados viram-se numa posição desfavorecida, incapazes de competir num mundo que parecia cada vez mais inclinado a favorecer os "geneticamente aprimorados".

   Deste modo, a disparidade no acesso à manipulação genética teve implicações profundas no âmbito da educação e das oportunidades de emprego. As crianças que tiveram acesso a intervenções genéticas desde cedo, muitas vezes provinientes de famílias
    mais abastadas, gozaram de vantagens significativas em termos de capacidades cognitivas e físicas. Este desequilíbrio sentiu-se profundamente nos sistemas educativos e no mercado de trabalho, criando ainda uma maior separação entre classes 
    económicas. 

   \subsection{Discriminação}

    As pessoas que optavam por manter a sua integridade genética era alvo de exclusão social e de discriminação.  Os \textit{GeneShifters} olhavam para eles como seres inferiores e irresponsáveis, comparando-os muitas das vezes com aqueles que eram negacionistas no que respeita
    aos benefícios das vacinações e também aqueles que por causa da religião negavam transforções de sangue quando estava em causa a sua vida e a dos seus familiares.

    Havia uma divisão entre a corrente negacionista em oposição à corrente filosófica do transumanismo.
    
    O mundo que já por si se encontrava dividido passou a separar aqueles que acreditavam nos benefícios da ciência daqueles que entendiam que a ciência não poderia ultrapassar certos limites. Tal facto levou designadamente a uma maior discriminação das minorias religiosas.

    Por outro lado, a ausência de deficiências tornou-se o modelo de normalidade, o que levou ao preconceito contra pessoas com alguma deficiência.

   \subsection{Resultados de uma Operação Mal Realizada}

    Nesta época surgiram humanos na qual a operação de \textit{gene editing} deu desastradamente errada. Qualquer alteração aos genes evoluí rapidamente. Um procedimento mal executado acarreta consequências gravíssimas. Jiankiu teve sorte quando alterou geneticamente
    as gêmeas. Contudo, com o aumento do uso do método que ele desenvolveu, era inevitável que ocorresserm falhas aqui e a ali. Isto resultou em diversos indivíduos com as suas vidas completamente arruínadas.

    Estas tragédias humanas tornaram-se o eco silencioso de um avanço científico que, apesar das suas promessas, não estava isento de riscos. 

    Como consequência, os \textit{PurityKeepers} levantaram as suas vozes apontando para cada tragédia como prova de que a humanidade estava a brincar com forças perigosas. Sempre que uma vida era afetada os jornais enchiam-se de declarações dos \textit{PurityKeepers},
     tais como: 
    
    \begin{center}
      \textit{"Cada tragédia é um alerta claro de que não podemos controlar completamente a natureza! Estamos a brincar de Deus e o preço é alto demais!"} \\
    \end{center}

    \begin{figure}[!htb]
      \centering
      \includegraphics[width=0.5\linewidth]{assets/protest.png}
      \caption{\textit{PurityKeepers} a protestar.}
      \label{fig:my_image}
    \end{figure}
    
    Uma outra preoucupação que este grupo levanta é a seguinte: a introdução de mudanças genéticas numa população, ainda que bem sucedida, pode levar a um impacto ecológico totalmente imprevisto. Implementar uma mutação genética para combater uma doença, tal como a malária, representa um risco considerável.
    Uma vez libertada a mutação vai-se espalhar conforme planeado e pode não ser revogada ou facilmente desativada. A própia mutação genética pode de alguma forma afetar uma espécie inofensiva causando danos incalculáveis ao ecossistema que sustenta
    a agricultura e outras formas de vida vegetal.

   \subsection{Dilemas éticos}

    Do lado dos \textit{PurityKeepers}, surgiam questões e preoucupações como:
    
    \begin{itemize}
      \item Deveriam as intervenções ser permitidas apenas por razões preventivas, diagnósticas e terapêuticas, sem realizar modificações para as gerações futuras?
      \item Estará a manipulação genética a comprometer a singularidade e a diversidade da experiência humana ao moldar o genoma de embriões sem o seu consentimento?
      \item Até que ponto a intervenção no genoma de embriões para fins reprodutivos deveria ser proibida de forma absoluta?
      \item Estarão os potenciais riscos a longo prazo a ser devidamente monitorizados? Será que as novas combinações hereditárias representam um risco imprevisível, semelhante a uma "lotaria" genética?
    \end{itemize}

    Do lado dos \textit{GeneShifters}, as posições eram distintas:

    \begin{itemize}
      \item A perspetiva de erradicar doenças genéticas traz consigo a promessa de poupanças significativas nos cuidados de saúde e um aumento notável na esperança de vida, com repercussões positivas na economia.
      \item A capacidade de melhorar as adaptações da nossa espécie a desafios ambientais e de saúde emergentes surge como uma oportunidade valiosa.
      \item Proporcionar a possibilidade de aprimorar características individuais pode resultar numa vida mais plena e satisfatória para todos os membros da sociedade.
      \item Os tratamentos altamente personalizados oferecem soluções mais eficazes para diversas condições médicas complexas.
    \end{itemize}

    Dentro deste turbilhão de debates e dilemas éticos, a humanidade continuava a avançar.

  \section{Conclusão}

  Assim conclui-se a nossa história. Contudo, pensamos que foram várias as questões que se levantaram, sendo uma delas, principalmente, 
  a seguinte:
  
  \begin{center}
    \textbf{Será que estamos preparados para uma revolução na área da Genética?}
  \end{center}

    Para iniciar esta discussão, é importante salientar que a história de He Jiankui que aqui apresentamos é, em parte, real.

    He Jiankui foi de facto um cientista que anunciou em 2018 experimentos repudiáveis que resultaram no nascimento dos primeiros três bebês geneticamente modificados. O suposto objetivo de He Jiankui era conferir
   imunidade ao HIV às crianças, uma intervenção perigosa e totalmente desnecessária, dado que os filhos de mães com HIV já nascem livres do vírus graças aos medicamentos antirretrovirais.

   Este escândalo conduziu à criação da Comissão Internacional sobre o Uso Clínico da Edição Genômica da Linha Germinal Humana.

  \begin{figure}[!htb]
    \centering
    \includegraphics[width=0.5\linewidth]{assets/He_Jiankui.jpg}
    \caption{\textit{He Jiankui, o cientista.}}
    \label{fig:my_image}
  \end{figure}

  Atualmente, sabemos que é possível modificar, eliminar e reorganizar o DNA de praticamente todos os organismos vivos. A tecnologia CRISPR é utilizada em todo o mundo para corrigir falhas genéticas significativas, 
   incluindo mutações responsáveis pela distrofia muscular, fibrose quística e uma forma de hepatite. No entanto, é importante mencionar que esta tecnologia ainda é relativamente dispendiosa. No âmbito da 
   nossa abordagem escolhemos imaginar um mundo onde tal não acontecia.

  Recentemente, a República Popular da China reviu a sua legislação no que concerne à modificação genética, permitindo alterações ao ADN com o intuito de imunizar contra determinadas doenças em seres humanos.

  Além disso, países como a Inglaterra, Argentina e os Estados Unidos da América uniram-se ao regulamentar o desenvolvimento de plantas geneticamente editadas.

  Consideramos essencial que exista um amplo debate social no que diz respeito ao uso da edição hereditária do genoma humano e aos limites para a sua aplicação.

  Esta prática não deve ser orientada pela filosofia do transumanismo para evitar a discriminação baseada no patrimônio genético de uma pessoa e prevenir profundas desigualdades sociais.

  O nosso trabalho visa alertar para esta realidade da genética e para os seus perigos quando utilizada de forma inadequada.

  No fundo, a manipulação do genoma humano é uma espada de dois gumes, capaz de curar mas também de prejudicar. É crucial que avancemos com sabedoria e responsabilidade, assegurando que cada passo dado seja guiado pela
   ética e pelo bem-estar da humanidade como um todo. Este é um apelo à ação para que todos nós, enquanto sociedade, participemos neste diálogo crucial sobre o futuro da edição genética. Juntos, podemos moldar um destino 
   onde a ciência e a humanidade caminhem lado a lado, promovendo a saúde e a igualdade para todos os seres humanos.
  
  

  \newpage

  \begin{thebibliography}{9} % O argumento indica o número máximo de entradas que você espera ter. Se você tem mais de 9, use 99.

    \bibitem{ExemploRef}
    Frontier Technology Quarterly (2019). \textit{Playing with Genes: the Good, the Bad and the Ugly}, \url{https://www.un.org/development/desa/dpad/publication/frontier-technology-quarterly-may-2019/}

    \bibitem{ExemploRef}
    Observador (2023). \textit{China altera lei sobre modificação genética. Especialistas estão preocupados}, \url{https://observador.pt/2023/03/08/china-altera-lei-sobre-modificacao-genetica-especialistas-estao-preocupados/}

    \bibitem{ExemploRef}
    CNN (2023). \textit{Controversial Chinese scientist He Jiankui proposes new gene editing research}, \url{https://edition.cnn.com/2023/07/03/china/he-jiankui-gene-editing-proposal-china-intl-hnk-scn/index.html}

    \bibitem{ExemploRef}
    Encyclopedia of Applied Ethics (Second Edition), P.D. Hopkins (2012). \textit{Transhumanism}

    \bibitem{ExemploRef}
    The Conversation (2018). \textit{We’re not prepared for the genetic revolution that’s coming}, \url{https://theconversation.com/were-not-prepared-for-the-genetic-revolution-thats-coming-96574}
    
  \end{thebibliography}


\end{document}