\documentclass[12pt]{article}
\usepackage[paper=letterpaper,margin=2cm]{geometry}
\usepackage{fancyhdr}
\usepackage{graphicx}
\usepackage{blindtext}
\usepackage{indentfirst}
\usepackage{listings}
\usepackage[colorlinks=true]{hyperref}
\usepackage[portuguese]{babel}

\date{\today}


\hypersetup{
    colorlinks=true,
    linkcolor=blue,
    filecolor=magenta,
    urlcolor=blue,
    citecolor=blue,
    pdftitle={Raquel Braunschweig - Análise Crítica 2023/2024},
    pdfpagemode=FullScreen,
}

\pagestyle{fancy}
\pagenumbering{arabic}
\fancyhf{}
\lhead{Mundos Alternativos - 2023/2024}
\rhead{Raquel Braunschweig (102624)}
\rfoot{\thepage}


\begin{document}
  
\begin{titlepage}
  \begin{center}

    \vspace*{5cm}
    \Huge
    Revisitando a Obra de Thomas More no Contexto Contemporâneo

    \vspace{0.5cm}
    \large
    28 de outubro de 2023

    \vspace*{2cm}
    \includegraphics[width=0.4\textwidth]{assets/IST-logo.png}~\\[2cm]


    \vspace*{1cm}
    Trabalho realizado, no âmbito da cadeira de Mundos Alternativos, por:
    
    \vspace{0.6cm}
    Raquel Braunschweig - 102624
  \end{center}
\end{titlepage}

  A obra de Thomas More, intitulada "Utopia", foi publicada em 1516 e descreve uma sociedade idealizada numa ilha fícticia.

  \newpage

  \begin{thebibliography}{9} % O argumento indica o número máximo de entradas que você espera ter. Se você tem mais de 9, use 99.

    \bibitem{Thome-More-Utopia}
    More, T. (1992). \textit{Utopia} (R. M. Adams, Trans.). Norton.
    
  \end{thebibliography}

\end{document}