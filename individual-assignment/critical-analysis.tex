\documentclass[12pt]{article}
\usepackage[paper=letterpaper,margin=2cm]{geometry}
\usepackage{fancyhdr}
\usepackage{graphicx}
\usepackage{blindtext}
\usepackage{indentfirst}
\usepackage{listings}
\usepackage[colorlinks=true]{hyperref}
\usepackage[portuguese]{babel}
\usepackage{parskip}
\usepackage{caption}

\date{\today}

\hypersetup{
    colorlinks=true,
    linkcolor=blue,
    filecolor=magenta,
    urlcolor=blue,
    citecolor=blue,
    pdftitle={Raquel Braunschweig - Análise Crítica 2023/2024},
    pdfpagemode=FullScreen,
}

\pagestyle{fancy}

\fancypagestyle{plain}{
    \fancyhf{} 
    \rfoot{\thepage} 
}

\lhead{Mundos Alternativos - 2023/2024}
\rhead{Raquel Braunschweig, (102624)}

\begin{document}

  
\begin{titlepage}
  \begin{center}

    \vspace*{5cm}
    \Huge
    Revisitando a Obra de Thomas More no Contexto Contemporâneo

    \vspace{0.5cm}
    \large
    28 de outubro de 2023

    \vspace*{2cm}
    \includegraphics[width=0.4\textwidth]{assets/IST-logo.png}~\\[2cm]


    \vspace*{1cm}
    Trabalho realizado, no âmbito da cadeira de Mundos Alternativos, por:
    
    \vspace{0.6cm}
    Raquel Braunschweig (Grupo 1) - 102624
  \end{center}
\end{titlepage}

A "Utopia" de Thomas More, publicada em 1516, descreve uma sociedade idealizada situada numa ilha fictícia. Esta obra, apesar de antiga, continua a ser relevante atualmente, notavelmente por apresentar críticas a certos aspetos da sociedade que ainda hoje persistem. Neste texto, irei analisá-la num contexto contemporâneo.$^{[1]}$

Primeiramente, More oferece uma análise incisiva sobre a corrupção e o abuso de poder, atribuindo a sua origem à instituição da propriedade privada. Ele acredita que numa sociedade utópica, a propriedade seria coletiva e o uso da moeda seria abolido.$^{[1]}$$^{[2]}$

Na sociedade contemporânea, as questões da corrupção e do abuso de poder permanecem uma preocupação global. Podemos ver isso nos vários escândalos políticos e empresariais que têm vindo a abalar a confiança dos cidadãos nas Instituições e levado ao aumento de movimentos extremistas e que se dizem “contra o sistema”.

Desde modo, \textbf{será que a mudança para uma sociedade comunal poderá ser a chave para nos aproximar-nos de um mundo utópico?} 

A ideia de propriedade coletiva e a abolição do uso da moeda desafia as normas políticas, sociais e económicas atuais. 

Existem, no entanto, milhares de comunidades comunitárias pelo mundo$^{[3]}$. Estas comunidades demonstram de facto uma menor corrupção, melhorias na qualidade de vida, maior senso de segurança e promoção do bem-estar coletivo.$^{[4]}$ Contudo, também enfrentam desafios que não podem ser ignorados. Por um lado, a falta de motivação revela-se problemática, sem uma estrutura clara de incentivos. As sociedades comunitárias por vezes têm dificuldades em motivar os indivíduos a trabalhar e a contribuir para a sociedade. Por outro lado, essas comunidades são criadas com base em princípios e objetivos partilhados. Deste modo, desacordos sobre como os recursos são alocados ou decisões são tomadas levam facilmente à fragmentação dessas comunidades.$^{[5]}$

A implementação deste estilo de sociedade em grande escala, além das dificuldades previamente expostas, também se depara com preocupações crescentes na contemporaneidade, nomeadamente a crise climática e a escassez de recursos para acomodar os 7.888 bilhões de habitantes do planeta. Segundo as Nações Unidas nos últimos 60 anos pelo menos 40\% dos conflitos estão ligados a recursos naturais.$^{[6]}$$^{[7]}$

Interpretar soluções diretas para a questão ambiental, à luz do trabalho de More, é complexo. No século XVI, a tecnologia era incipiente e a crise climática não detinha a urgência que ostenta hoje. 

More sugere na Utopia como solução para as cidades sobrepovoadas deslocar um grupo de indivíduos a se estabeleceram na terra desabitada mais próxima$^{[1]}$$^{[6]}$.

Esta ideia, todavia, não se amolda à realidade contemporânea. Os centros populacionais estão dispersos globalmente, não deixando áreas habitáveis para um estabelecimento desse tipo. Isto culminaria em conflito e desestabilização na região, possivelmente em violação do direito internacional.

O drama dos refugiados climáticos e dos fluxos migratórios ilegais e descontrolados reflete bem a dificuldade no acolhimento destes.$^{[8]}$

\begin{minipage}{1\textwidth}
    \centering
    \includegraphics[width=0.6\linewidth]{assets/refugiados-climaticos.jpg}
    \captionof{figure}{Refugiados cilmáticos do Haiti à procura de asilo}
\end{minipage}

Por outro lado, More salienta que numa sociedade utópica todos os indivíduos seriam educados. Esta ideia, apesar de aparentemente distante da crise ambiental, desempenha um papel crucial. Com efeito, para que uma sociedade possa prevalecer frente aos desafios e equívocos relativos à casa comum é imperativo que todos os seus membros - e não apenas os legisladores - sejam instruídos sobre a questão ambiental e simultaneamente que sejam providenciadas as ferramentas para conter a voracidade com que o mundo se esgota. Programas educacionais dessa natureza representam uma das respostas a tais questões. Neste sentido, é crucial valorizarmos a ideia de More. $^{[1]}$$^{[2]}$$^{[6]}$

Nos dias atuais, o impacto de Thomas More não passa despercebido. Na comunidade de capitalistas do Silicon Valley o debate acerca da construção de uma sociedade perfeita ainda é um tema atual. Levemos como exemplo Peter Thiel que já terá investido mais de 1 milhão de dólares num projeto chamado \textit{Seasteading Institute} e que tem como missão a seguinte: “Estabelecer permanentes e autónomas comunidades no meio do oceano de modo a poder experimentar diversos sistemas políticos, sociais e legais”. Este projeto evoca algo familiar, não é mesmo?$^{[9]}$

Em suma, a análise das ideias utópicas de Thomas More continua a fornecer uma fonte inesgotável de inspiração e reflexão sobre a sociedade ideal. A publicação da sua obra "Utopia" ofereceu uma visão profunda acerca das questões da corrupção, da igualdade de género e da importância da educação, temas que ecoam na contemporaneidade. Além disso, ao desafiar as convenções sociais e económicas do seu tempo, More instiga-nos a reavaliar e a reimaginar o funcionamento da nossa própria sociedade.

Preservar tais visões utópicas não é apenas um exercício intelectual, mas uma necessidade urgente. Elas recordam-nos de que a imaginação é uma ferramenta poderosa na busca por um mundo melhor. Ao cultivarmos a capacidade de idealizar e conceber sociedades mais justas, equitativas e sustentáveis, damos os primeiros passos em direção à concretização desses ideais.


  \newpage

  \begin{thebibliography}{10} % O argumento indica o número máximo de entradas que você espera ter. Se você tem mais de 9, use 99.

    \bibitem{Thome-More-Utopia}
    More, T. (1992). \textit{Utopia} (R. M. Adams, Trans.). Norton.

    \bibitem{Thome-More-Ideas}
    Bedore P. (2021). \textit{Thomas More's Utopian Ideas}, https://www.wondriumdaily.com/thomas-mores-utopian-ideas/

    \bibitem{Communal Living}
    Reece E. (2016). \textit{Utopia Now}, https://www.theatlantic.com/business/archive/2016/08/utopia-erik-reece/494741/

    \bibitem{Communal Living - Positives}
    Webster A. (2022). \textit{The benefits of communal living}, https://www.perspectivesmag.co.uk/post/the-benefits-of-communal-living

    \bibitem{Communal Living - Negatives}
    Nomic N. (2019). \textit{What are some reasons why communal societies fail to thrive in the long term?}, https://www.quora.com/What-are-some-reasons-why-communal-societies-fail-to-thrive-in-the-long-term

    \bibitem{Environment And Utopia}
    Corman D. (2018). \textit{Utopia and Contemporary Human Society: A Model for Sustainable Continuance}. University of Nebraska.

    \bibitem{UN}
    United Nations (2019). \textit{Conflict and Natural Resources}. https://peacekeeping.un.org/en/conflict-and-natural-resources

    \bibitem{Parlamento}
    Parlamento Europeu (2023). \textit{Migração: um desafio comum}. https://www.europarl.europa.eu/news/pt/headlines/priorities/migracao

    \bibitem{Parlamento}
    Hodgkinson (2016). \textit{How Utopia shaped the world}. https://www.bbc.com/culture/article/20160920-how-utopia-shaped-the-world

  \end{thebibliography}

\end{document}